\documentclass[10pt, a4paper]{article}
\usepackage[utf8]{inputenc}

% enumerate con letras
\usepackage[shortlabels]{enumitem}

\usepackage{bm} % \bm, https://www.ctan.org/pkg/bm

% Para hacer derivaciones y eso
% https://personal.utdallas.edu/~hamlen/trfrac.pdf
\usepackage{tfrac}

% Simbolos matematicos
\usepackage{mathtools}
\usepackage{amsfonts}

% https://www.overleaf.com/learn/latex/page_size_and_margins
\usepackage{geometry}
\geometry{
    a4paper,
    left=13pt
}

% Lambda calculo
% Robado amablemente de
% https://github.com/CubaWiki/PLP-resumen-iarcuschin/blob/master/resumen.tex#L61
\newcommand{\abs}[3]{\ensuremath{\lambda#1:#2.#3}}
\newcommand{\app}[2]{\ensuremath{#1\ #2}}
\newcommand{\sust}[2]{\ensuremath{#1\{#2\}}}
\newcommand{\eqdef}[0]{\ \stackrel{\mathclap{\normalfont\mbox{def}}}{=}\ }
\newcommand{\ifte}[3]{\ensuremath{if\ #1\ then\ #2\ else\ #3}}
\newcommand{\tipa}[3]{#1 \ \rhd \ #2 : #3}

\newcommand{\Gtipa}[2]{\tipa{\Gamma}{#1}{#2}}
\newcommand{\GStipa}[2]{\ensuremath{\Gamma|\Sigma \rhd #1 : #2}}

%\newcommand{\deriv}[3]{\ensuremath{\frac{#1}{#2}\ \text{(#3)}}}
%\renewcommand{\vert}{\ |\ }
%\newcommand{\from}[0]{\leftarrow}
%\newcommand{\toto}[0]{\twoheadrightarrow}
%\newcommand{\erase}[1]{\textsc{Erase}(\ensuremath{#1})}
%\newcommand{\stabs}[2]{\ensuremath{\lambda#1.#2}}
%\newcommand{\w}[1]{\ensuremath{\mathbb{W}(#1)}}
%\newcommand{\uni}[0]{\ \stackrel{\mathclap{\normalfont\mbox{.}}}{=}\ }
%\newcommand{\unito}[0]{\mapsto}


% Macros especificos al ejercicio
\newcommand{\MAT}[1]{\text{MAT}(#1)}
\newcommand{\MatInf}[1]{\text{MatInf}_{#1}}
\newcommand{\MatGet}[3]{#1[#2][#3]}
\newcommand{\MatSet}[4]{\MatGet{#1}{#2}{#3} \leftarrow #4}

\begin{document}

%% trfrac
% \trfrac[name]{premises}{consequent}
%
% Ejemplo
%
% \[ \trfrac[\small(\emph{app})]
% {\Gamma \vdash e_1:\tau\rightarrow\tau' \qquad
% \trfrac[\small(\emph{app})]
% {\Gamma \vdash e_2:\tau''\rightarrow\tau \qquad
% \Gamma \vdash e_3:\tau''}
% {\Gamma \vdash e_2 e_3:\tau}}
% {\Gamma \vdash e_1 (e_2 e_3):\tau'} \]

\section*{Ej 2}

\begin{enumerate}[a)]

\item \textit{Reglas de tipado}

Agrego una regla de tipado para cada termino agregado a la extension

\[ \trfrac[(T-MAT)]{\Gtipa{M}{\sigma}}{\Gtipa{\MAT{M}}{\MatInf{\sigma}}} \]

\[
    \trfrac[(T-MatSet)]
        {
            \Gtipa{M}{\MatInf{\sigma}} \qquad 
            \Gtipa{P}{\sigma} \qquad
            \Gtipa{N, O}{Nat}
        }
        {
            \Gtipa{\MatSet{M}{N}{O}{P}}{\MatInf{\sigma}}
        }
\]

\[
    \trfrac[(T-MatGet)]
        {
            \Gtipa{M}{\MatInf{\sigma}} \qquad 
            \Gtipa{N, O}{Nat}
        }
        {
            \Gtipa{\MatGet{M}{N}{O}}{\sigma}
        }
\]

\[
    \trfrac[(T-MatMap)]
        {
            \Gtipa{M}{\MatInf{\sigma}} \qquad 
            \Gtipa{N}{\sigma \rightarrow \tau}
        }
        {
            \Gtipa{map(M, N)}{\MatInf{\tau}}
        }
\]


\item \textit{Demostrar el siguiente juicio de tipado o explicar por que no es derivable
}
\[
    \tipa
        {\{i: Nat, m: \MatInf{Nat}\}}
        {map((\MatSet{m}{i}{i}{0}), \abs{x}{Nat}{isZero(x)})}
        {\MatInf{Bool}}
\]

Intuitivamente deberia ser derivable, ya que mapea una matriz infinita de
naturales a una matriz de booleanos que indica si cada uno era cero. Intento de
demostrarlo

\newcommand{\deriv}[3]{\trfrac[(\emph{#1})]{#2}{#3}}

\[
    \trfrac[(T-MatMap)]
    {
        \trfrac[(T-MatSet)]
        {
            \begin{trgather}
            \trfrac[(T-Var)]
            {
                \trfrac[]
                {\checkmark}
                {
                    m:\MatInf{Nat} \in \Gamma
                }
            }
            {
                \Gtipa{m}{\MatInf{Nat}}
            }
            \\
            \trfrac[(T-Var)]
            {
                \trfrac[]
                {\checkmark}
                {
                    i:Nat \in \Gamma
                }
            }
            {
                \Gtipa{i}{Nat}
            }
            \qquad\qquad\qquad\quad
            \trfrac[(T-Zero)]
            {\checkmark}
            {
                \Gtipa{0}{Nat}
            }
            \end{trgather}
        }
        {
            \Gtipa{\MatSet{m}{i}{i}{0}}{\MatInf{Nat}}
        }
        \quad
        \trfrac[(T-Abs)]
        {
            \trfrac[(T-IsZero)]
            {
                \trfrac[(T-Var)]
                {
                    \trfrac[]
                    {\checkmark}
                    {
                        x:Nat \in \Gamma \cup \{x:Nat\}   
                    }
                }
                {
                    \tipa{\Gamma \cup \{x:Nat\}}{x}{Nat}
                }
            }
            {
                \tipa{\Gamma \cup \{x:Nat\}}{isZero(x)}{Bool}
            }
        }
        {
            \Gtipa{\abs{x}{Nat}{isZero(x)}}{Nat \rightarrow Bool}
        }
    }
    {
        \tipa
        {\Gamma = \{i: Nat, m: \MatInf{Nat}\}}
        {map((\MatSet{m}{i}{i}{0}), \abs{x}{Nat}{isZero(x)})}
        {\MatInf{Bool}}
    } 
\]

\item \textit{Indicar formalmente cómo se modifica el conjunto de valores, o explicar
por qué no se modifica.}

    Es necesario agregar las matrices como valor, la idea va a ser similar a como se
    construyen los naturales. $MAT(M)$ va a ser un valor con una matriz con todos
    Ms, y se va construir con $\MatSet{V}{X}{Y}{M}$. Esto quiere decir que, por
    ejemplo, $\MatSet{(\MatSet{(\MatSet{MAT(0)}{0}{0}{1})}{1}{1}{2})}{2}{2}{3}$ seria un
    valor valido. El conjunto de valores entonces sería

    \[
        V, X, Y ::= \ldots \mid MAT(M) \mid \MatSet{V}{X}{Y}{M}
    \]

\item \textit{Dar la semántica operacional de a un paso para la extensión.}


\[
    \trfrac[(E-MatSetC)] {
        M_1 \to M_1'
    }{
        \MatSet{M_1}{X}{Y}{M_2} \to \MatSet{M_1'}{X}{Y}{M_2}
    }
\]
\[
    \trfrac[(E-MatSetXC)] {
        N \to N'
    }{
        \MatSet{M_1}{N}{O}{M_2} \to \MatSet{M_1}{N'}{O}{M_2}
    }
\]
\[
    \trfrac[(E-MatSetYC)] {
        O \to O'
    }{
        \MatSet{M_1}{N}{O}{M_2} \to \MatSet{M_1}{N}{O'}{M_2}
    }
\]
\[
    \trfrac[(E-MatGetNoMatch)]{
    }{
        \MatGet{(\MatSet{V}{I}{J}{M})}{X}{Y} \to \MatGet{V}{X}{Y}
    }
\]
\[
    \trfrac[(E-MatGetXC)]{
        N \to N'
    }{
        \MatGet{M}{N}{O} \to \MatGet{M}{N'}{O}
    }
\]
\[
    \trfrac[(E-MatGetYC)]{
        O \to O'
    }{
        \MatGet{M}{X}{O} \to \MatGet{M}{X}{O'}
    }
\]
\[
    \trfrac[(E-MatGetC)]{
        M \to M'
    }{
        \MatGet{M}{X}{Y} \to \MatGet{M'}{X}{Y}
    }
\]
\[
    \trfrac[(E-MatGetSetC)]{
        M \to M'
    }{
        \MatGet{(\MatSet{V}{X}{Y}{M})}{X}{Y} \to \MatGet{(\MatSet{V}{X}{Y}{M'})}{X}{Y}
    }
\]
\[
    \trfrac[(E-MatGetMATC)]{
        M \to M'
    }{
        \MatGet{(\MAT{M})}{X}{Y} \to \MatGet{(\MAT{M'})}{X}{Y}
    }
\]
\[
    \trfrac[(E-MatGetMAT)]{
    }{
        \MatGet{(\MAT{V})}{X}{Y} \to V
    }
\]
\[
    \trfrac[(E-MatGetSet)]{
    }{
        \MatGet{(\MatSet{V}{X}{Y}{W})}{X}{Y} \to W
    }
\]
\[
    \trfrac[(E-MatMapCF)]{
        F \to F'
    }{
        map(M, F) \to map(M, F')
    }
\]
\[
    \trfrac[(E-MatMapCM)]{
        M \to M'
    }{
        map(M, F) \to map(M', F)
    }
\]
\[
    \trfrac[(E-MatMapMAT)]{
    }{
        map(MAT[M], F) \to MAT(F M)
    }
\]
\[
    \trfrac[(E-MatMapSet)]{
    }{
        map(\MatSet{V}{X}{Y}{M}, F) \to \MatSet{map(V, F)}{X}{Y}{F\ M}
    }
\]

TODO: para matcheo de indices no vale poner la misma letra, hay que preguntar por igualdad sintactica con $\equiv$, como dice el enunciado.

\item \textit{Mostrar como se reducen paso a paso}

\begin{enumerate}[I)]
    \item $\MatGet{ (\MatSet{MAT(\abs{y}{Bool}{y})}{succ(0)}{0}{\abs{x}{Bool}{\text{true}}}) }{0}{succ(0)}$
    
    \begin{align*}
        &\MatGet{ (\MatSet{MAT(\abs{y}{Bool}{y})}{succ(0)}{0}{\abs{x}{Bool}{\text{true}}}) }{0}{succ(0)}\\
        &\to_\text{(E-MatGetNoMatch)}
            \MatGet{(MAT(\abs{y}{Bool}{y})) }{0}{succ(0)}\\
        &\to_\text{(E-MatGetMAT)}
            \abs{y}{Bool}{y}
    \end{align*}

    \item $\MatGet{( (\abs{x}{Bool}{\MatSet{\MAT{x}}{0}{0}{false}})\ true)}{0}{pred(succ(0)}$
    \begin{align*}
        &\MatGet{( (\abs{x}{Bool}{\MatSet{\MAT{x}}{0}{0}{false}})\ true)}{0}{pred(succ(0)}\\
        &\to_\text{(E-MatGetYC), (E-PredSucc)}
            \MatGet{( (\abs{x}{Bool}{\MatSet{\MAT{x}}{0}{0}{false}})\ true)}{0}{\bm{0}}\\
        &\to_\text{(E-MatGetC, E-AppAbs)}
            \MatGet{(\MatSet{\MAT{\bm{true}}}{0}{0}{false})}{0}{\bm{0}}\\
        &\to_\text{(E-MatGetSet)} false
    \end{align*}

    \item $\MatGet{map(\MatSet{\MAT{0}}{0}{0}{succ(0)}, \abs{x}{Nat}{isZero(x)})}{0}{0}$
    \begin{align*}
        &\MatGet{map(\MatSet{\MAT{0}}{0}{0}{succ(0)}, \abs{x}{Nat}{isZero(x)})}{0}{0}\\
        &\to_\text{(E-MatGetC, E-MatMapSet)}
            \MatGet{\MatSet{map(\MAT{0}, \abs{x}{Nat}{isZero(x)})}{0}{0}{(\abs{x}{Nat}{isZero(x)})\ succ(0)})}{0}{0}\\
        &\to_\text{(E-MatGetC, E-MatMapMAT)}
            \MatGet{\MatSet{\MAT{(\abs{x}{Nat}{isZero(x)})\ 0}}{0}{0}{(\abs{x}{Nat}{isZero(x)})\ succ(0)})}{0}{0}\\
        &\to_\text{(E-MatGetSetC, E-AppAbs)}
            \MatGet{\MatSet{\MAT{(\abs{x}{Nat}{isZero(x)})\ 0}}{0}{0}{isZero(succ(0))})}{0}{0}\\
        &\to_\text{(E-MatGetSetC, E-IsZeroSucc)}
            \MatGet{\MatSet{\MAT{(\abs{x}{Nat}{isZero(x)})\ 0}}{0}{0}{\bm{false}})}{0}{0}\\
        &\to_\text{(E-MatGetSet, E-IsZeroSucc)} false
    \end{align*}

\end{enumerate}

\end{enumerate}
\end{document}
